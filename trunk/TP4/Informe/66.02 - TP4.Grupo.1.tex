\documentclass{article}

%% PAQUETES

% Paquetes generales
\usepackage[margin=2cm, paperwidth=210mm, paperheight=297mm]{geometry}
\usepackage[spanish]{babel}
\usepackage[utf8]{inputenc}
\usepackage{gensymb}

% Paquetes para estilos
\usepackage{textcomp}
\usepackage{setspace}
\usepackage{colortbl}
\usepackage{color}
\usepackage{color}
\usepackage{upquote}
\usepackage{xcolor}
\usepackage{listings}
\usepackage{caption}
\usepackage[T1]{fontenc}
\usepackage[scaled]{beramono}

% Paquetes extras
\usepackage{amssymb}
\usepackage{float}
\usepackage{graphicx}
\usepackage{array}
\usepackage{multirow}
\usepackage{amsmath}

%% Fin PAQUETES


% Definición de preferencias para la impresión de código fuente.
%% Colores
\definecolor{gray99}{gray}{.99}
\definecolor{gray95}{gray}{.95}
\definecolor{gray75}{gray}{.75}
\definecolor{gray50}{gray}{.50}
\definecolor{keywords_blue}{rgb}{0.13,0.13,1}
\definecolor{comments_green}{rgb}{0,0.5,0}
\definecolor{strings_red}{rgb}{0.9,0,0}

%% Caja de código
\DeclareCaptionFont{white}{\color{white}}
\DeclareCaptionFont{style_labelfont}{\color{black}\textbf}
\DeclareCaptionFont{style_textfont}{\it\color{black}}
\DeclareCaptionFormat{listing}{\colorbox{gray95}{\parbox{16.78cm}{#1#2#3}}}
\captionsetup[lstlisting]{format=listing,labelfont=style_labelfont,textfont=style_textfont}

\lstset{
	aboveskip = {1.5\baselineskip},
	backgroundcolor = \color{gray99},
	basicstyle = \ttfamily\footnotesize,
	breakatwhitespace = true,   
	breaklines = true,
	captionpos = t,
	columns = fixed,
	commentstyle = \color{comments_green},
	escapeinside = {\%*}{*)}, 
	extendedchars = true,
	frame = lines,
	keywordstyle = \color{keywords_blue}\bfseries,
	language = Octave,                       
	numbers = left,
	numbersep = 5pt,
	numberstyle = \tiny\ttfamily\color{gray50},
	prebreak = \raisebox{0ex}[0ex][0ex]{\ensuremath{\hookleftarrow}},
	rulecolor = \color{gray75},
	showspaces = false,
	showstringspaces = false, 
	showtabs = false,
	stepnumber = 1,
	stringstyle = \color{strings_red},                                    
	tabsize = 2,
	title = \null, % Default value: title=\lstname
	upquote = true,                  
}

%% FIGURAS
\captionsetup[figure]{labelfont=bf,textfont=it}
%% TABLAS
\captionsetup[table]{labelfont=bf,textfont=it}

% COMANDOS

%% Titulo de las cajas de código
\renewcommand{\lstlistingname}{Código}
%% Titulo de las figuras
\renewcommand{\figurename}{Figura}
\addto\captionsspanish{\renewcommand{\figurename}{Figura}}
%% Titulo de las tablas
\renewcommand{\tablename}{Tabla}
\addto\captionsspanish{\renewcommand{\tablename}{Tabla}}
%% Referencia a los códigos
\newcommand{\refcode}[1]{\textit{Código \ref{#1}}}
%% Referencia a las imagenes
\newcommand{\refimage}[1]{\textit{Imagen \ref{#1}}}



\begin{document}


% OBJETIVOS
\section{Objetivos}

	El objetivo del trabajo práctico es la familiarización con el uso de las puntas del osciloscopio, tanto en X1 como en X10, además de los controles más complejos del mismo, tales como la base de tiempo secundaria, barrido alternado, choppeado, etc. Por último, se espera adquirir una especial destreza en la realización de mediciones más complejas.
\bigskip\bigskip




% INTRODUCCIÓN
\section{Introducción}
\medskip

% INTRODUCCIÓN - Puntas
\subsection{Puntas}

	El componente más crítico de un sistema de medida basado en un osciloscopio es su propia punta; la calidad de la medición siempre estará limitada por la calidad de la sonda. Su elección correcta deberá considerar no sólo las especificaciones del osciloscopio sino también  las del circuito bajo prueba y las características de la señal a medir.
	\par
	Las sondas se fabrican con componentes pasivos (resistencias, inductores y capacitores) que habrá que tener en cuenta por el efecto de carga al sistema que pueden llegar a provocar. Para que esta incerteza sea despreciable se busca que

\begin{equation*}
	R_{circ} \ll R_{op}
\end{equation*}
\begin{equation*}
	C_{circ} \gg C_{op}
\end{equation*}
\medskip

	También existe otra especificación para una punta pasiva: su factor de atenuación. Este determina la proporción que hay entre las amplitudes de las señales de entrada y salida. Cuanto más elevado es, menor es la sensibilidad vertical del sistema de medida punta-osciloscopio. Sin embargo, la ventaja de las puntas atenuadoras radica en reducir la carga eléctrica del sistema de medida sobre el circuito a medir.
\bigskip



% INTRODUCCIÓN - Tiempo de crecimiento de una señal
\subsection{Tiempo de crecimiento de una señal}

	Sabemos que cuando se aplica una tensión a un circuito RC, la carga del capacitor demandará cierto tiempo. El retraso en el crecimiento de la tensión sobre un capacitor puede ponerse de manifiesto a través del parámetro llamado tiempo de crecimiento. Para una onda cuadrada, se define a esta variable como el tiempo que le lleva a la señal aumentar desde el 10\% al 90\% de su tensión máxima, y se calcula mediante la fórmula:

\begin{equation*}
	T_c = 2,2 \times RC
\end{equation*}
\smallskip


% INTRODUCCIÓN - Frecuencia de corte
\subsection{Frecuencia de corte}

Definimos como frecuencia de corte a la frecuencia para la cual la respuesta en frecuencia cae al 70,7\% de su valor máximo (se reduce en un valor de 3dB), es decir

\begin{equation*}
	V_0 = {V_i \over \sqrt{2}}
\end{equation*}
\medskip

\noindent En un circuito RC, esta frecuencia se obtiene según

\begin{equation*}
	f_c = {1 \over 2 \pi RC}
\end{equation*}
\medskip

\bigskip\bigskip




% MATERIALES UTILIZADOS
\section{Materiales utilizados}

	Se detallan a continuación (\textit{Tabla 1}) la lista de materiales y dispositivos utilizados durante el desarrollo de la práctica, acompañados por sus respectivas características y especificaciones principales. Para más información sobre el instrumental puede dirijirse a la sección \textit{Apéndice A}, ubicada al final del presente informe, donde se adjuntan las hojas de datos de todos estos.
\bigskip\bigskip


% Tabla 1
\begin{table}[!hbt]
	\begin{center}
	\begin{tabular}{|>{\centering\arraybackslash}m{5cm}|>{\arraybackslash}m{6cm}|}
		\hline
		\rowcolor[gray]{0.9}\textbf{Material/Instrumento} & \textbf{Especificaciones} \\
		\hline
		Generador de funciones & Modelo: 8140\\
		\hline
		Osciloscipio & \vbox{\hbox{\strut Marca: GOOD-WILL }
						   \hbox{\strut Modelo: 653G }}\\
		\hline
		Contador & \vbox{\hbox{\strut Marca: GOOD-WILL }
						   \hbox{\strut Modelo: guc-2020 }}\\
		\hline
		Cables & Banana-Cocodrilo\newline Cocodrilo-Cocodrilo\newline BNC-BNC\newline Banana-BNC \\
		\hline
	\end{tabular}
	\caption{Listado de materiales e instrumental utilizado.}
	\end{center}
\end{table}
\bigskip\bigskip




% DESARROLLO
\section{Desarrollo}

	En los siguientes apartados se pasarán a desarrollar las mediciones empíricas, cada una de las cuales esta complementada con una explicación de los pasos llevados a cabo, valores obtenidos, análisis de resultados y conclusiones parciales.
\bigskip



% DESARROLLO - Medicion del tiempo de crecimiento
\subsection{Medición del tiempo de crecimiento}
	Se dispuso del siguiente banco de medición:\\
	\bigskip
	
	[Insertar la figura!]\\
	\bigskip
	
	Inicialmente, se calculó la frecuencia de corte y el tiempo de crecimiento del circuito \textit{RC} de manera teórica, y sin tener en cuenta el efecto de carga que producen las puntas y los instrumentos de medición. Como los valores de los elementos que se utilizaron son \textbf{$C = 68pF$} y \textbf{$R = 1k\Omega $}, entonces:

\begin{equation*}
	f_c = {1 \over 2 \pi RC} = {1 \over 2 \pi \times 1k\Omega \times 68pF} = 2,34 Mhz
\end{equation*}
\medskip
Y el tiempo de crecimiento es:
\begin{equation*}
	T_c = 2,2 \times RC = 2,2 \times 1k\Omega \times 68pF = 149,6 ns
\end{equation*}

En la práctica el efecto de carga es imposible de evitar, y se midió el tiempo de crecimiento  y la frecuencia de corte con los dos tipos de puntas disponibles, la \textit{x1}, y la \textit{x10}.
El procedimiento para ambos fue el mismo:\\
\bigskip

Para el tiempo de crecimiento, se utilizó el \textit{CH 2} del osciloscopio (que es el que mide la caída de tensión en el capacitor) y se midió el tiempo que le toma a la señal pasar del 10\% al 90\%. La exactitud en la sección horizontal proporcionada por el fabricante es del 3\% de la medida, más otro 3\% por linealidad.\\

Con respecto a la frecuencia de corte, se buscó que ambas señales tuviesen un desfasaje de \textit{45°} , que es en el momento en que se encuentra en dicha frecuencia de corte.
El método utilizado fue calcular el período de la señal, y luego medir el tiempo de desfase entre ambas señales, y viendo la relación.
 
\subsubsection{Medición con la punta \textit{x1}}
Los valores medidos fueron:




\subsubsection{Medición con la punta \textit{x10}} 
Los valores medidos fueron:


% DESARROLLO - Respuesta en frecuencia
\subsection{Medición de la respuesta en frecuencia}
	
	[ Colocar texto aquí ]
	\bigskip



% DESARROLLO - Determinación de la frecuencia de corte
\subsection{Determinación de la frecuencia de corte}
	
	[ Colocar texto aquí ]
	\bigskip



% DESARROLLO - Rectificadores
\subsection{Rectificadores}
	
	[ Colocar texto aquí ]
	\bigskip




% CONCLUSIONES
\section{Conclusiones}

	De acuerdo a los resultados obtenidos en apartados anteriores podemos concluir que el efecto de carga que introducen las puntas en circuitos RC puede ser considerable tanto usando la punta \textit{X1} como la \textit{X10}. Esto se confirma al ver que los tiempos de crecimiento de las señales eran apreciablemente distintos de los calculados analíticamente. Aún así se puede ver que la punta atenuadora \textit{X10} es la mejor opción para realizar el trabajo práctico. Al ser el capacitor de $68 pF$, no hay punta que mejore las medidas realizadas mucho más, porque hay que tener en cuenta la capacidad de entrada del osciloscopio, que no se puede despreciar. 
	\par
	Se pudo observar también la relación directa entre el ancho de banda de los circuitos con el tiempo de crecimiento, y los valores utilizados de resistencias y capacidades. 
	\par
	Finalmente analizamos la utilización de diodos como rectificadores de media onda y onda completa, pudiendo así deducir los factores de forma.
\bigskip\bigskip


\newpage \textit{}
\newpage



% APÉNDICE A
\newpage
\vspace*{4cm}
\begin{center}
	\textbf{\Huge{Apéndice A}} \\
	\bigskip\bigskip
	\Large{\textit{``Hojas de datos de instrumentos de medición''}}
\end{center}


\newpage \textit{}
\newpage

\end{document}
