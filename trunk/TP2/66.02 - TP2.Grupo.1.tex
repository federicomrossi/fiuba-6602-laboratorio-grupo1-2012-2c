\documentclass{article}

%% PAQUETES

% Paquetes generales
\usepackage[margin=2cm, paperwidth=210mm, paperheight=297mm]{geometry}
\usepackage[spanish]{babel}
\usepackage[utf8]{inputenc}
\usepackage{gensymb}

% Paquetes para estilos
\usepackage{textcomp}
\usepackage{setspace}
\usepackage{colortbl}
\usepackage{color}
\usepackage{color}
\usepackage{upquote}
\usepackage{xcolor}
\usepackage{listings}
\usepackage{caption}
\usepackage[T1]{fontenc}
\usepackage[scaled]{beramono}

% Paquetes extras
\usepackage{amssymb}
\usepackage{float}
\usepackage{graphicx}
\usepackage{array}
\usepackage{multirow}

%% Fin PAQUETES


% Definición de preferencias para la impresión de código fuente.
%% Colores
\definecolor{gray99}{gray}{.99}
\definecolor{gray95}{gray}{.95}
\definecolor{gray75}{gray}{.75}
\definecolor{gray50}{gray}{.50}
\definecolor{keywords_blue}{rgb}{0.13,0.13,1}
\definecolor{comments_green}{rgb}{0,0.5,0}
\definecolor{strings_red}{rgb}{0.9,0,0}

%% Caja de código
\DeclareCaptionFont{white}{\color{white}}
\DeclareCaptionFont{style_labelfont}{\color{black}\textbf}
\DeclareCaptionFont{style_textfont}{\it\color{black}}
\DeclareCaptionFormat{listing}{\colorbox{gray95}{\parbox{16.78cm}{#1#2#3}}}
\captionsetup[lstlisting]{format=listing,labelfont=style_labelfont,textfont=style_textfont}

\lstset{
	aboveskip = {1.5\baselineskip},
	backgroundcolor = \color{gray99},
	basicstyle = \ttfamily\footnotesize,
	breakatwhitespace = true,   
	breaklines = true,
	captionpos = t,
	columns = fixed,
	commentstyle = \color{comments_green},
	escapeinside = {\%*}{*)}, 
	extendedchars = true,
	frame = lines,
	keywordstyle = \color{keywords_blue}\bfseries,
	language = Octave,                       
	numbers = left,
	numbersep = 5pt,
	numberstyle = \tiny\ttfamily\color{gray50},
	prebreak = \raisebox{0ex}[0ex][0ex]{\ensuremath{\hookleftarrow}},
	rulecolor = \color{gray75},
	showspaces = false,
	showstringspaces = false, 
	showtabs = false,
	stepnumber = 1,
	stringstyle = \color{strings_red},                                    
	tabsize = 2,
	title = \null, % Default value: title=\lstname
	upquote = true,                  
}

%% FIGURAS
\captionsetup[figure]{labelfont=bf,textfont=it}
%% TABLAS
\captionsetup[table]{labelfont=bf,textfont=it}

% COMANDOS

%% Titulo de las cajas de código
\renewcommand{\lstlistingname}{Código}
%% Titulo de las figuras
\renewcommand{\figurename}{Figura}
\addto\captionsspanish{\renewcommand{\figurename}{Figura}}
%% Titulo de las tablas
\renewcommand{\tablename}{Tabla}
\addto\captionsspanish{\renewcommand{\tablename}{Tabla}}
%% Referencia a los códigos
\newcommand{\refcode}[1]{\textit{Código \ref{#1}}}
%% Referencia a las imagenes
\newcommand{\refimage}[1]{\textit{Imagen \ref{#1}}}



\begin{document}


% OBJETIVOS
\section{Objetivos}

	El objetivo del trabajo práctico es familiarizarse con el uso de los diferentes multímetros utilizados como voltímetros, y predecir las alteraciones que trae su uso, conociendo sus especificaciones.
\bigskip



% INTRODUCCIÓN
\section{Introducción}

	[ Colocar texto aquí ]
\bigskip




% MATERIALES UTILIZADOS
\section{Materiales utilizados}

	Se detallan a continuación (\textit{Tabla 1}) la lista de materiales y dispositivos utilizados durante el desarrollo de la práctica, acompañados por sus respectivas características y especificaciones principales. Para más información sobre el instrumental puede dirijirse a la sección \textit{Apéndice B}, ubicada al final del presente informe, donde se adjuntan las hojas de datos de todos estos.
\bigskip\bigskip


% Tabla 1
\begin{table}[!hbt]
	\begin{center}
	\begin{tabular}{|>{\centering\arraybackslash}m{5cm}|>{\arraybackslash}m{6cm}|}
		\hline
		\rowcolor[gray]{0.9}\textbf{Material/Instrumento} & \textbf{Especificaciones} \\
		\hline
		\centering Resistencias &  \vbox{\hbox{\strut 100$\Omega\pm5\%$ tolerancia (1 unidad)}
						    \hbox{\strut 100k$\Omega\pm5\%$ tolerancia (2 unidades)}
						    \hbox{\strut 10M$\Omega\pm5\%$ tolerancia (1 unidad)}} \\
		\hline
		Multímetro analógico & \vbox{\hbox{\strut Marca: TRIPLETT }
						    \hbox{\strut Modelo: 630-APLK }
						    \hbox{\strut Alcance: 5000V }
						    \hbox{\strut Sensibilidad: 20k$\Omega$/V}
						    \hbox{\strut Incerteza de clase: 3,5\%}
						    \hbox{\strut Impedancia de entrada: 200k$\Omega$}}\\
		\hline
		Multímetro digital & \vbox{\hbox{\strut Marca: UNI-T }
						    \hbox{\strut Modelo: UT30F }
						    \hbox{\strut Alcance: 500V }
						    \hbox{\strut Incerteza: 0,5\%}
						    \hbox{\strut Impedancia de entrada: 10M$\Omega$}}\\
		\hline
		Multímetro digital & \vbox{\hbox{\strut Marca: Brymen }
						    \hbox{\strut Modelo: BM837RS }}\\
		\hline
		Cables & Banana-Cocodrilo\newline Cocodrilo-Cocodrilo \\
		\hline
	\end{tabular}
	\caption{Listado de materiales e instrumental utilizado.}
	\end{center}
\end{table}




% DESARROLLO
\section{Desarrollo}

	En los siguientes apartados se pasarán a desarrollar las mediciones empíricas, cada una de las cuales esta complementada con una explicación de los pasos llevados a cabo, valores obtenidos, análisis de resultados y conclusiones parciales.
\bigskip



% DESARROLLO - PARTE A
\subsection{Parte A}

En esta parte del trabajo práctico, se hicieron distintas mediciones con el fin de comparar las diferentes formas que tienen los multímetros (en modo voltímetro) de calcular los valores de tensión.\\
\indent Para esto, se utilizó un osciloscopio, un generador de funciones, una resistencia de 47 $\Omega$ y tres multímetros diferentes: uno analógico, uno de valor medio, y uno de valor medio verdadero.\\
\indent El circuito armado consitió en el generador de funciones, la resistencia y el osciloscopio conectados en paralelo. También en paralelo se fueron conectando los multímetros para realizar las mediciones. \\
\indent Se ajustó el generador para que el osciloscopio muestre una señal con un duty cicle del \textbf{30\%},
con \textbf{5V} de amplitud, y con una frecuencia de \textbf{100 Hz} , para asegurarnos de que esté dentro del ancho de banda de los instrumentos.\\
\indent Las incertidumbres de los instrumentos no fueron tomadas en cuenta, ya que son irrelevantes para el objetivo de esta parte de la prácitca.
\medskip

\subsubsection{Multímetro DVM}

En \textbf{DC}, el multímetro de valor medio mide justamente el valor medio de la tensión presente.\\

En \textbf{AC}, tiene un capacitor en la entrada, por lo que elimina el valor continuo de la tensión. Además rectifica la señal por onda completa, y calcula el valor medio y lo multiplica por el factor de forma, 1,11(en onda completa), para obtener el valor eficaz.\\
\bigskip

\subsubsection{Multímetro True-RMS}

En \textbf{DC}, el multímetro de valor medio verdadero mide el valor medio de la tensión presente.\\

En \textbf{AC}, tiene un capacitor en la entrada, por lo que elimina el valor continuo de la tensión. Además rectifica la señal por onda completa, y calcula el valor eficaz de la tensión alterna. Luego calcula el valor eficaz verdadero, mediante la siguiente ecuación:

\begin{center}
\begin{equation}
V = \sqrt{ {{1}\over{T}} \times \int_{0}^{T} {V(t)}^2 \delta t}
\end{equation}
\end{center}
\bigskip


\subsubsection{Multímetro Analógico}

En \textbf{DC}, el multímetro analógico mide el valor medio de la tensión presente.\\

En \textbf{AC} (utilzado con la salida \textit{output}), realiza el mismo proceso que el multímetro DVM.

\newpage
% Tabla 2
\begin{table}[!hbt]
	\begin{center}

		\begin{tabular}{|c|c|c|c|c|c|c|c|} \hline
			\multirow{4}{*}{\textbf{Tipo de onda}}

			& \multicolumn{6}{c|}{\textbf{Mediciones}} \\\cline{2-7}
			& \multicolumn{2}{c|}{\textbf{VOM}} & \multicolumn{2}{c|}{\textbf{DVM}} & \multicolumn{2}{c|}{\textbf{TRUE}} \\\cline{2-7}
			& \multicolumn{2}{c|}{\textbf{V}} & \multicolumn{2}{c|}{\textbf{V}} & \multicolumn{2}{c|}{\textbf{V}} \\\cline{2-7}
			& \textbf{DC} & \textbf{AC} & \textbf{DC} & \textbf{AC} & \textbf{DC} & \textbf{AC} \\\hline
			\textbf{Cuadrada} & -2,00 & 4,7  & -2,00 & 4,66 & -2,000 & 4,66 \\\hline
			\textbf{Senoidal} & 0 & 3,2 & 0 & 3,53 & 0 &  3,536 \\\hline
			\textbf{Triangular} & 0 & 2,50 & 0 & 2,50 & 0 & 3,536 \\\hline
		\end{tabular}

	\caption{Tabla de valores calculados.}
	\end{center}
\end{table}
\bigskip


% Tabla 3
\begin{table}[!hbt]
	\begin{center}

		\begin{tabular}{|c|c|c|c|c|c|c|c|} \hline
			\multirow{4}{*}{\textbf{Tipo de onda}}

			& \multicolumn{6}{c|}{\textbf{Mediciones}} \\\cline{2-7}
			& \multicolumn{2}{c|}{\textbf{VOM}} & \multicolumn{2}{c|}{\textbf{DVM}} & \multicolumn{2}{c|}{\textbf{TRUE}} \\\cline{2-7}
			& \multicolumn{2}{c|}{\textbf{V}} & \multicolumn{2}{c|}{\textbf{V}} & \multicolumn{2}{c|}{\textbf{V}} \\\cline{2-7}
			& \textbf{DC} & \textbf{AC} & \textbf{DC} & \textbf{AC} & \textbf{DC} & \textbf{AC} \\\hline
			\textbf{Cuadrada} & -1,95 & 5,0 & -2,05 & 4,47 & -2,009 & 4,35 \\\hline
			\textbf{Senoidal} & 0,05 & 3,00 & 0,042 & 3,45 & 0,301 & 3,446 \\\hline
			\textbf{Triangular} & 0,00 & 2,40 & 0,017 & 2,64 & 0,587 & 2,778 \\\hline
		\end{tabular}

	\caption{Tabla de valores medidos.}
	\end{center}
\end{table}
\bigskip



% DESARROLLO - PARTE B
\subsection{Parte B}

En esta sección del trabajo práctico se comprobó la respuesta en frecuencias de tres tipos distintos de multímetros, es decir, el ancho de banda para el cual las mediciones de los multímetros caen dentro del rango de incertidumbre propuesto por el fabricante.


Para realizar las mediciones, se dispuso del mismo circuito que en la \textbf{Parte A}, y con los distintos voltímetro midiendo la salida del generador en paralelo. Además se conectó el circuito a un contador, para una medida más precisa de la frecuencia.

En una primera instancia, se midió un valor de tensión de referencia, con una frecuencia dentro de las establecidas por los fabricantes. De este modo se determinó un valor de \textbf{$V_{máx}$} y de \textbf{$V_{min}$}, y los valores medidos dentro de este rango se consideraron como medidas válidas. A partir de esto, se paso a determinar el ancho de banda de los instrumentos.
\bigskip

Se utilizaron 3 tipos de voltímetros:
\bigskip

\begin{table}[!hbt]
	\begin{center}
	\begin{tabular}{|c|c|}\hline
	\textbf{Voltímetro} & \textbf{Ancho de banda(manual)}\\ \hline
    DVM &  (40 - 400)$Hz$\\ \hline
    True - RMS &  (100 - 100k)$Hz$\\ \hline
    Analógico &  (10 - ???)$Hz$\\ \hline
	\end{tabular}
	\caption{Tabla de los anchos de banda provistos por el fabricante.}
	\end{center}
\end{table}

\subsubsection{Multímetro DVM}

La incertidumbre para la escala utilizada es:

\begin{equation}
 	\Delta(V) = 1\%\times V_{medido} + 3d\times 10mV,
\end{equation}
\medskip

Los valores medidos son:

\begin{center}
$V_{ref} = (3.39 \pm 0.04) V$
\end{center}

\begin{table}[!hbt]
	\begin{center}
	\begin{tabular}{|c|c|c|}\hline
	\textbf{f(Hz)} & \textbf{V(Volts)} & \textbf{$\Delta$V(Volts)} \\ \hline

	30 & 3.44 &  0.06	\\ \hline
    40 & 3.43 &	0.06\\ \hline
    50 & 3.43 &	0.06\\ \hline
	210 & 3.41 & 0.06\\ \hline
	220 & 3.40 & 0.06\\ \hline
	230 & 3.40 & 0.06\\ \hline
	390 & 3.39 & 0.06\\ \hline
	400 & 3.39 & 0.06\\ \hline
	410 & 3.39 & 0.06\\	 \hline
	500 & 3.38 & 0.06\\ \hline
	600 & 3.38 & 0.06\\ \hline
	\end{tabular}
	\caption{Tabla de los valores medidos con el voltímetro DVM}
	\end{center}
\end{table}
\bigskip


\begin{figure}[h!tbp]
\centering
\includegraphics[width=0.80\textwidth]{images/tablaDVM.png}
\caption{Gráfico de los valores medidos con el voltímetro DVM.}
\end{figure}



\subsubsection{Multímetro True-RMS}


La incertidumbre para la escala utilizada es:

\begin{equation}
 	\Delta(V) = 1\%\times V_{medido} + 3d\times 1mV.
\end{equation}
\medskip

Los valores medidos son:

\begin{center}
$V_{ref} = (3.420 \pm 0.037) V$
\end{center}

\begin{table}[!hbt]
	\begin{center}
	\begin{tabular}{|c|c|c|}\hline
	\textbf{f(Hz)} & \textbf{V(Volts)} & \textbf{$\Delta$V(Volts)} \\ \hline
	80 & 3.417 & 0.037\\ \hline
    90 & 3.418 & 0.037\\ \hline
    100 & 3.419 & 0.037\\ \hline
	110 & 3.423 & 0.037\\ \hline
	440 & 3.419 & 0.037\\ \hline
	450 & 3.420 & 0.037\\ \hline
	460 & 3.420 & 0.037\\ \hline
	990 & 3.393 & 0.037\\ \hline
	1000 & 3.398 & 0.037\\ \hline
	1010& 3.396 & 0.037\\ \hline
	1500 & 3.385 & 0.037\\ \hline
	1700 & 3.379 & 0.037\\ \hline
	\end{tabular}
	\caption{Tabla de los valores medidos con el voltímetro True-RMS}
	\end{center}
\end{table}
\bigskip

\begin{figure}[h!tbp]
\centering
\includegraphics[width=0.80\textwidth]{images/tablaTRUE.png}
\caption{Gráfico de los valores medidos con el voltímetro True-RMS.}
\end{figure}
\bigskip

\subsubsection{Multímetro Analógico}


La incertidumbre para la escala utilizada es:

\begin{equation}
 	\varepsilon_r(V) = 3\%\times V_{medido} + {1 \over 2}\times 0.2V
\end{equation}
\medskip

Valores medidos:

\begin{center}
$V_{ref} = (3.4 \pm 0.2) V$
\end{center}

\begin{table}[!hbt]
	\begin{center}
	\begin{tabular}{|c|c|c|}\hline
	\textbf{f(Hz)} & \textbf{V(Volts)} & \textbf{$\Delta$V(Volts)} \\ \hline
	100 & 3.4 & 0.2\\ \hline
    213000 & 3.2 & 0.2\\ \hline
    758000 & 3.0 & 0.2\\ \hline
	\end{tabular}
	\caption{Tabla de los valores obtenidos con el voltímetro analógico}
	\end{center}
\end{table}
\bigskip


\begin{figure}[h!tbp]
\centering
\includegraphics[width=0.80\textwidth]{images/tablaANA.png}
\caption{Gráfico de los valores medidos con el voltímetro analógico.}
\end{figure}
\bigskip

Por lo tanto, los valores aproximados de los anchos de banda son:

\begin{table}[!hbt]
	\begin{center}
	\begin{tabular}{|c|c|}\hline
	\textbf{Voltímetro} & \textbf{Ancho de banda(experimentales)}\\ \hline
    DVM &  $\sim$ (30 - 900)$Hz$\\ \hline
    True - RMS &  $\sim$ (30 - 1500)$Hz$\\ \hline
    Analógico & $\sim$ (10 - 213000)$Hz$\\ \hline
	\end{tabular}
	\caption{Tabla de los anchos de banda experimentales}
	\end{center}
\end{table}



% APÉNDICE A

\newpage
\vspace*{4cm}
\begin{center}
	\textbf{\Huge{Apéndice A}} \\
	\bigskip\bigskip
	\Large{\textit{``Simulaciones''}}
\end{center}


\newpage \textit{}
\newpage

	En el presente apéndice se muestran los resultados obtenidos al simular los circuitos propuestos en la \textit{sección X.X}. Dichas simulaciones fueron realizadas con el software \textit{Proteus}\footnote{Para mas información sobre el software \textit{Proteus} puede dirijirse a \textit{http://www.labcenter.com/index.cfm}}.
\bigskip \bigskip




% APÉNDICE B

\newpage
\vspace*{4cm}
\begin{center}
	\textbf{\Huge{Apéndice B}} \\
	\bigskip\bigskip
	\Large{\textit{``Hojas de datos de instrumentos de medición''}}
\end{center}

\end{document}
