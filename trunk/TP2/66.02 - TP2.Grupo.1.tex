\documentclass{article}

%% PAQUETES

% Paquetes generales
\usepackage[margin=2cm, paperwidth=210mm, paperheight=297mm]{geometry}
\usepackage[spanish]{babel}
\usepackage[utf8]{inputenc}
\usepackage{gensymb}

% Paquetes para estilos
\usepackage{textcomp}
\usepackage{setspace}
\usepackage{colortbl}
\usepackage{color}
\usepackage{color}
\usepackage{upquote}
\usepackage{xcolor}
\usepackage{listings}
\usepackage{caption}
\usepackage[T1]{fontenc}
\usepackage[scaled]{beramono}

% Paquetes extras
\usepackage{amssymb}
\usepackage{float}
\usepackage{graphicx}
\usepackage{array}
\usepackage{multirow}

%% Fin PAQUETES


% Definición de preferencias para la impresión de código fuente.
%% Colores
\definecolor{gray99}{gray}{.99}
\definecolor{gray95}{gray}{.95}
\definecolor{gray75}{gray}{.75}
\definecolor{gray50}{gray}{.50}
\definecolor{keywords_blue}{rgb}{0.13,0.13,1}
\definecolor{comments_green}{rgb}{0,0.5,0}
\definecolor{strings_red}{rgb}{0.9,0,0}

%% Caja de código
\DeclareCaptionFont{white}{\color{white}}
\DeclareCaptionFont{style_labelfont}{\color{black}\textbf}
\DeclareCaptionFont{style_textfont}{\it\color{black}}
\DeclareCaptionFormat{listing}{\colorbox{gray95}{\parbox{16.78cm}{#1#2#3}}}
\captionsetup[lstlisting]{format=listing,labelfont=style_labelfont,textfont=style_textfont}

\lstset{
	aboveskip = {1.5\baselineskip},
	backgroundcolor = \color{gray99},
	basicstyle = \ttfamily\footnotesize,
	breakatwhitespace = true,   
	breaklines = true,
	captionpos = t,
	columns = fixed,
	commentstyle = \color{comments_green},
	escapeinside = {\%*}{*)}, 
	extendedchars = true,
	frame = lines,
	keywordstyle = \color{keywords_blue}\bfseries,
	language = Octave,                       
	numbers = left,
	numbersep = 5pt,
	numberstyle = \tiny\ttfamily\color{gray50},
	prebreak = \raisebox{0ex}[0ex][0ex]{\ensuremath{\hookleftarrow}},
	rulecolor = \color{gray75},
	showspaces = false,
	showstringspaces = false, 
	showtabs = false,
	stepnumber = 1,
	stringstyle = \color{strings_red},                                    
	tabsize = 2,
	title = \null, % Default value: title=\lstname
	upquote = true,                  
}

%% FIGURAS
\captionsetup[figure]{labelfont=bf,textfont=it}
%% TABLAS
\captionsetup[table]{labelfont=bf,textfont=it}

% COMANDOS

%% Titulo de las cajas de código
\renewcommand{\lstlistingname}{Código}
%% Titulo de las figuras
\renewcommand{\figurename}{Figura}
\addto\captionsspanish{\renewcommand{\figurename}{Figura}}
%% Titulo de las tablas
\renewcommand{\tablename}{Tabla}
\addto\captionsspanish{\renewcommand{\tablename}{Tabla}}
%% Referencia a los códigos
\newcommand{\refcode}[1]{\textit{Código \ref{#1}}}
%% Referencia a las imagenes
\newcommand{\refimage}[1]{\textit{Imagen \ref{#1}}}



\begin{document}


% OBJETIVOS
\section{Objetivos}

	El objetivo del trabajo práctico es familiarizarse con el uso de los diferentes multímetros utilizados como voltímetros, y predecir las alteraciones que trae su uso, conociendo sus especificaciones.
\bigskip



% INTRODUCCIÓN
\section{Introducción}

	[ Colocar texto aquí ]
\bigskip




% MATERIALES UTILIZADOS
\section{Materiales utilizados}

	Se detallan a continuación (\textit{Tabla 1}) la lista de materiales y dispositivos utilizados durante el desarrollo de la práctica, acompañados por sus respectivas características y especificaciones principales. Para más información sobre el instrumental puede dirijirse a la sección \textit{Apéndice B}, ubicada al final del presente informe, donde se adjuntan las hojas de datos de todos estos.
\bigskip\bigskip


% Tabla 1
\begin{table}[!hbt]
	\begin{center}
	\begin{tabular}{|>{\centering\arraybackslash}m{5cm}|>{\arraybackslash}m{6cm}|}
		\hline
		\rowcolor[gray]{0.9}\textbf{Material/Instrumento} & \textbf{Especificaciones} \\
		\hline
		\centering Resistencias &  \vbox{\hbox{\strut 100$\Omega\pm5\%$ tolerancia (1 unidad)}
						    \hbox{\strut 100k$\Omega\pm5\%$ tolerancia (2 unidades)}
						    \hbox{\strut 10M$\Omega\pm5\%$ tolerancia (1 unidad)}} \\
		\hline
		Multímetro analógico & \vbox{\hbox{\strut Marca: TRIPLETT }
						    \hbox{\strut Modelo: 630-APLK }
						    \hbox{\strut Alcance: 5000V }
						    \hbox{\strut Sensibilidad: 20k$\Omega$/V}
						    \hbox{\strut Incerteza de clase: 3,5\%}
						    \hbox{\strut Impedancia de entrada: 200k$\Omega$}}\\
		\hline
		Multímetro digital & \vbox{\hbox{\strut Marca: UNI-T }
						    \hbox{\strut Modelo: UT30F }
						    \hbox{\strut Alcance: 500V }
						    \hbox{\strut Incerteza: 0,5\%}
						    \hbox{\strut Impedancia de entrada: 10M$\Omega$}}\\
		\hline
		Multímetro digital & \vbox{\hbox{\strut Marca: Brymen }
						    \hbox{\strut Modelo: BM837RS }}\\
		\hline
		Cables & Banana-Cocodrilo\newline Cocodrilo-Cocodrilo \\
		\hline
	\end{tabular}
	\caption{Listado de materiales e instrumental utilizado.}
	\end{center}
\end{table}




% DESARROLLO
\section{Desarrollo}

	En los siguientes apartados se pasarán a desarrollar las mediciones empíricas, cada una de las cuales esta complementada con una explicación de los pasos llevados a cabo, valores obtenidos, análisis de resultados y conclusiones parciales.
\bigskip



% DESARROLLO - PARTE A
\subsection{Parte A}

\newpage
% Tabla 2
\begin{table}[!hbt]
	\begin{center}

		\begin{tabular}{|c|c|c|c|c|c|c|c|} \hline
			\multirow{4}{*}{\textbf{Tipo de onda}}

			& \multicolumn{6}{c|}{\textbf{Mediciones}} \\\cline{2-7}
			& \multicolumn{2}{c|}{\textbf{VOM}} & \multicolumn{2}{c|}{\textbf{DVM}} & \multicolumn{2}{c|}{\textbf{TRUE}} \\\cline{2-7}
			& \multicolumn{2}{c|}{\textbf{V}} & \multicolumn{2}{c|}{\textbf{V}} & \multicolumn{2}{c|}{\textbf{V}} \\\cline{2-7}
			& \textbf{DC} & \textbf{AC} & \textbf{DC} & \textbf{AC} & \textbf{DC} & \textbf{AC} \\\hline
			\textbf{Cuadrada} &  &  &  &  &  &  \\\hline
			\textbf{Senoidal} &  &  &  &  &  &  \\\hline
			\textbf{Triangular} &  &  &  &  &  &  \\\hline
		\end{tabular}

	\caption{Tabla de valores calculados.}
	\end{center}
\end{table}
\bigskip


% Tabla 3
\begin{table}[!hbt]
	\begin{center}

		\begin{tabular}{|c|c|c|c|c|c|c|c|} \hline
			\multirow{4}{*}{\textbf{Tipo de onda}}

			& \multicolumn{6}{c|}{\textbf{Mediciones}} \\\cline{2-7}
			& \multicolumn{2}{c|}{\textbf{VOM}} & \multicolumn{2}{c|}{\textbf{DVM}} & \multicolumn{2}{c|}{\textbf{TRUE}} \\\cline{2-7}
			& \multicolumn{2}{c|}{\textbf{V}} & \multicolumn{2}{c|}{\textbf{V}} & \multicolumn{2}{c|}{\textbf{V}} \\\cline{2-7}
			& \textbf{DC} & \textbf{AC} & \textbf{DC} & \textbf{AC} & \textbf{DC} & \textbf{AC} \\\hline
			\textbf{Cuadrada} & 1,95 & 5,0 & 2,049 & 4,47 & 2,01 & 4,35 \\\hline
			\textbf{Senoidal} & 0,05 & 3,0 & 0,042 & 3,45 & 0,30 & 3,45 \\\hline
			\textbf{Triangular} & 0,00 & 2,4 & 0,017 & 2,64 & 0,59 & 2,74 \\\hline
		\end{tabular}

	\caption{Tabla de valores medidos.}
	\end{center}
\end{table}
\bigskip





% DESARROLLO - PARTE B
\subsection{Parte B}

\subsubsection{Multímetro DVM}

Incertidumbre:

\begin{equation}
 	\Delta(V) = 1\%\times V_{medido} + 3d\times 10mV,
\end{equation}
\medskip


\begin{table}[!hbt]
	\begin{center}
	\begin{tabular}{|c|c|c|}\hline
	\textbf{f(Hz)} & \textbf{V(Volts)} & \textbf{$\Delta$V(Volts)} \\ \hline

	30 & 3.44 &  0.06	\\ \hline
    40 & 3.43 &	0.06\\ \hline
    50 & 3.43 &	0.06\\ \hline
	210 & 3.41 & 0.06\\ \hline
	220 & 3.40 & 0.06\\ \hline
	230 & 3.40 & 0.06\\ \hline
	390 & 3.39 & 0.06\\ \hline
	400 & 3.39 & 0.06\\ \hline
	410 & 3.39 & 0.06\\	 \hline
	500 & 3.38 & 0.06\\ \hline
	600 & 3.38 & 0.06\\ \hline
	100000 & 1.91 &	0.05\\ \hline
	\end{tabular}
	\caption{Tabla del voltímetro DVM}
	\end{center}
\end{table}
\bigskip


\subsubsection{Multímetro True-RMS}

Incertidumbre:

\begin{equation}
 	\Delta(V) = 1\%\times V_{medido} + 3d\times 1mV.
\end{equation}
\medskip


\begin{table}[!hbt]
	\begin{center}
	\begin{tabular}{|c|c|c|}\hline
	\textbf{f(Hz)} & \textbf{V(Volts)} & \textbf{$\Delta$V(Volts)} \\ \hline
	80 & 3.417 & 0.04\\ \hline
    90 & 3.418 & 0.04\\ \hline
    100 & 3.419 & 0.04\\ \hline
	110 & 3.423 & 0.04\\ \hline
	440 & 3.419 & 0.04\\ \hline
	450 & 3.420 & 0.04\\ \hline
	460 & 3.420 & 0.04\\ \hline
	990 & 3.393 & 0.04\\ \hline
	1000 & 3.398 & 0.04\\ \hline
	1010& 3.396 & 0.04\\ \hline
	1500 & 3.385 & 0.04\\ \hline
	100000 & 0.170 & 0.04\\ \hline
	\end{tabular}
	\caption{Tabla del voltímetro True-RMS}
	\end{center}
\end{table}
\bigskip

\subsubsection*{Multímetro Analógico}

Incertidumbre:

\begin{equation}
 	\varepsilon_r(V) = 3\%\times V_{medido} + {1 \over 2}\times 0.2V
\end{equation}
\medskip

\begin{table}[!hbt]
	\begin{center}
	\begin{tabular}{|c|c|c|}\hline
	\textbf{f(Hz)} & \textbf{V(Volts)} & \textbf{$\Delta$V(Volts)} \\ \hline
	100 & 3.4 & 0.2\\ \hline
    213000 & 3.2 & 0.2\\ \hline
    758000 & 3.0 & 0.2\\ \hline
	8.33M & 1.0 & 0.2\\ \hline
	\end{tabular}
	\caption{Tabla del voltímetro Analógico}
	\end{center}
\end{table}

% APÉNDICE A

\newpage
\vspace*{4cm}
\begin{center}
	\textbf{\Huge{Apéndice A}} \\
	\bigskip\bigskip
	\Large{\textit{``Simulaciones''}}
\end{center}


\newpage \textit{}
\newpage

	En el presente apéndice se muestran los resultados obtenidos al simular los circuitos propuestos en la \textit{sección X.X}. Dichas simulaciones fueron realizadas con el software \textit{Proteus}\footnote{Para mas información sobre el software \textit{Proteus} puede dirijirse a \textit{http://www.labcenter.com/index.cfm}}.
\bigskip \bigskip




% APÉNDICE B

\newpage
\vspace*{4cm}
\begin{center}
	\textbf{\Huge{Apéndice B}} \\
	\bigskip\bigskip
	\Large{\textit{``Hojas de datos de instrumentos de medición''}}
\end{center}

\end{document}
