\documentclass{article}

%% PAQUETES

% Paquetes generales
\usepackage[margin=2cm, paperwidth=210mm, paperheight=297mm]{geometry}
\usepackage[spanish]{babel}
\usepackage[utf8]{inputenc}
\usepackage{gensymb}

% Paquetes para estilos
\usepackage{textcomp}
\usepackage{setspace}
\usepackage{colortbl}
\usepackage{color}
\usepackage{color}
\usepackage{upquote}
\usepackage{xcolor}
\usepackage{listings}
\usepackage{caption}
\usepackage[T1]{fontenc}
\usepackage[scaled]{beramono}

% Paquetes extras
\usepackage{amssymb}
\usepackage{float}
\usepackage{graphicx}

%% Fin PAQUETES


% Definición de preferencias para la impresión de código fuente.
%% Colores
\definecolor{gray99}{gray}{.99}
\definecolor{gray95}{gray}{.95}
\definecolor{gray75}{gray}{.75}
\definecolor{gray50}{gray}{.50}
\definecolor{keywords_blue}{rgb}{0.13,0.13,1}
\definecolor{comments_green}{rgb}{0,0.5,0}
\definecolor{strings_red}{rgb}{0.9,0,0}

%% Caja de código
\DeclareCaptionFont{white}{\color{white}}
\DeclareCaptionFont{style_labelfont}{\color{black}\textbf}
\DeclareCaptionFont{style_textfont}{\it\color{black}}
\DeclareCaptionFormat{listing}{\colorbox{gray95}{\parbox{16.78cm}{#1#2#3}}}
\captionsetup[lstlisting]{format=listing,labelfont=style_labelfont,textfont=style_textfont}

\lstset{
	aboveskip = {1.5\baselineskip},
	backgroundcolor = \color{gray99},
	basicstyle = \ttfamily\footnotesize,
	breakatwhitespace = true,   
	breaklines = true,
	captionpos = t,
	columns = fixed,
	commentstyle = \color{comments_green},
	escapeinside = {\%*}{*)}, 
	extendedchars = true,
	frame = lines,
	keywordstyle = \color{keywords_blue}\bfseries,
	language = Octave,                       
	numbers = left,
	numbersep = 5pt,
	numberstyle = \tiny\ttfamily\color{gray50},
	prebreak = \raisebox{0ex}[0ex][0ex]{\ensuremath{\hookleftarrow}},
	rulecolor = \color{gray75},
	showspaces = false,
	showstringspaces = false, 
	showtabs = false,
	stepnumber = 1,
	stringstyle = \color{strings_red},                                    
	tabsize = 2,
	title = \null, % Default value: title=\lstname
	upquote = true,                  
}

%% FIGURAS
\captionsetup[figure]{labelfont=bf,textfont=it}
%% TABLAS
\captionsetup[table]{labelfont=bf,textfont=it}

% COMANDOS

%% Titulo de las cajas de código
\renewcommand{\lstlistingname}{Código}
%% Titulo de las figuras
\renewcommand{\figurename}{Figura}
%% Titulo de las tablas
\renewcommand{\tablename}{Tabla}
%% Referencia a los códigos
\newcommand{\refcode}[1]{\textit{Código \ref{#1}}}
%% Referencia a las imagenes
\newcommand{\refimage}[1]{\textit{Imagen \ref{#1}}}



\begin{document}

% Inserción del título, autores y fecha.
\title{\huge 66.02 Laboratorio \\ 
	  \Huge Trabajo Práctico N°1 \\
	  \bigskip \Large 10 de septiembre de 2012 \\
	  \bigskip \large \textbf{Grupo 1} \\
	  \large \textit{Scialabba, Julián (92181)\\Aguilera, Juan Martín (92483)\\Rossi, Federico Martín (92086)}}
\date{}
\maketitle



% OBJETIVOS
\section{Objetivos}

% INTRODUCCIÓN
\section{Introducción}

	Con el fin de fortalecer las nociones fundamentales de la probabilidad es que este trabajo busca aplicar las bases de los conceptos ya estudiados en el curso de manera tal de adoptar una mejor comprensión de la inferencia estadística en hechos cotidianos. En este caso, nuestro trabajo será explorar las propiedades estádisticas de cierto tipo de secuencias, las cuales serán detalladas más adelante.
	\par 
	En cada análisis realizado se mostrará su resolución analítica, como así también algoritmos que simulen dichas situaciones, a fin de poder mostrar una aproximación a la constatación empírica de los resultados. Estos últimos se han hecho en el lenguaje de programación \textit{GNU Octave}\footnote{``\textit{GNU Octave} es un lenguaje interpretado de alto nivel, pensado principalmente para cálculos numéricos. Para más información dirijase a \textit{http://www.gnu.org/software/octave} ''}.
	\par 
	Todos los archivos y códigos fuente aquí mencionados, así como también el presente informe, pueden ser descargados de la sección \textit{Downloads} del repositorio del grupo (\textit{http://code.google.com/p/simulacion-proba2012}).



% MATERIALES UTILIZADOS
\section{Materiales utilizados}

Tal como lo sugiere el título de este apartado, se mostrarán a continuación una serie de actividades previas necesarias para dominar las técnicas básicas de simulación de números aleatorios.




% DESARROLLO
\section{Desarrollo}

% DESARROLLO - PARTE 1
\subsection{Parte 1}



% DESARROLLO - PARTE 2
\subsection{Parte 2}



% DESARROLLO - PARTE 3
\subsection{Parte 3}


\end{document}
